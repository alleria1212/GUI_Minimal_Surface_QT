\chapter{Entwicklerdokumentation}
\label{ch:5}

\section{Codestruktur}
Der Code besteht aus mehreren Klassen. Die Quelldateien befinden sich gr\"o\ss tenteils im {\tt root} Verzeichnis. Die verwendeten Bibliotheken befinden sich in den Ordnern mit gleichem Namen. Siehe Kapitel 4 f\"ur Installation. Im Folgenden wird die Codestruktur im Groben beschrieben. F\"ur eine detaillierte Dokumentation siehe Abschnitt 2 in diesem Kapitel.

\subsection{Klasse MainWindow}
Die Klasse {\tt MainWindow} ist f\"ur die graphische Darstellung der Benutzeroberfl\"ache zust\"andig. Der Quellcode findet sich in \refsec{ssec:6.1.1} und in den Dateien {\tt mainwindow.h} und {\tt mainwindow.cpp}. Au\ss erdem wird in dieser Klasse das Laden und Speichern von Einstellung und Ergebnis definiert.

\subsection{Klasse Controller}
Die Klasse {\tt Controller} verwaltet die Berechnung des Ergebnisses. Sie steuert die Klasse {\tt Discretization}, welche auf einem anderen Thread rechnet, damit das Interface weiter verwendet werden kann.  Der Quellcode befindet sich in \refsec{6.1.2} und in den Dateien {\tt controller.h} und {\tt controller.cpp}.

\subsection{Klasse Paramter}
Die Klasse {\tt Parameter} ist komplett \"offentlich definiert. Es existiert zur Laufzeit immer maximal ein {\tt Parameter} Objekt welches per Referenz an alle Klassen vergeben, die dieses benötigen. Der Quellcode befindet sich in \refsec{6.1.3} und in den Dateien {\tt parameter.h} und {\tt parameter.cpp}.

\subsection{Klasse Discretization}
In der Klasse {\tt Discretization} ist die eigentliche Berechnung der Minimalfl\"ache unter Verwendung des Newton Verfahrens implementiert. Da {\tt Discretization} auf einem anderen Thread arbeitet als die anderen Klassen, existiert auch nur ein Objekt, welches erst nach der Beendung des Programms gel\"oscht wird. Der Quellcode befindet sich in \refsec{6.1.4} und in den Dateien {\tt discretization.h} und {\tt discretization.cpp}.

\subsection{Klasse Plot}
Die Klasse {\tt Plot} wird von der Klasse {\tt Qwt3D::SurfacePlot} abgeleitet, und stellt lediglich das Plot Widget zu Verf\"ugung. Der Quellcode befindet sich in \refsec{6.1.5} und in der Datei {\tt plot.h}.

\subsection{Hilfsfunktionen}
Die eigen definierten Hilfsfunktionen f\"ur den L\"oser befinden sich in \refsec{6.1.6} und in der Datei {\tt solverfunctions.h}

\section{Detaillierte Dokumentation des Codes}
In dem Unterverzeichnis {\tt ./minimalsurface/doc/} befindet sich die Konfigurationsdatei {\tt Doxyfile} f\"ur die Erstellung der doxygen Dokumentation in {\tt HTML} oder \LaTeX\ Format. Zum Erstellen der Dokumentation muss der Befehl {\tt doxygen Doxyfile} in dem Verzeichnis {\tt ./minimalsurface/doc/} aufgerufen werden. Es werden zwei neue Verzeichnisse {\tt ./minimalsurface/doc/html/} und {\tt ./minimalsurface/doc/latex/} erstellt. Um die {\tt HTML} Version anzuzeigen muss in den Ordner {\tt html} gewechselt werden und die Datei {\tt index.html} mit dem Standardbrowser ge\"offnet werden. Um die \LaTeX\ Dokumentation anzeigen zu lassen muss der Befehl {\tt make} in dem Ordner {\tt latex} ausgef\"uhrt werden. Anschlie\ss end kann die PDF Datei mit einem entsprechenden Programm gelesen werden.

\section{Software-Tests}
Alle Anwendungsf\"alle wurden manuell auf Funktionalit\"at getestet. Die folgenden Anwendungsf\"alle wurden erfolgreich ohne Fehlermeldungen getestet:
\begin{itemize}
	\item Beenden
	\item Tab wechseln
	\item Speichern
	\item Laden
	\item Einstellungen \"andern
\end{itemize}
Der Anwendungsfall Berechnung starten wurde erfolgreich mit allen Fehlersituationen getestet, die in der Benutzerdokumentation ausf\"uhrlicher beschrieben sind.