\textbf{Check Randfunktionen}
  \begin{itemize}
  \item \textit{Ziel:} Das System will die Randfunktionen \"uberpr\"ufen.
  \item \textit{Einordnung:} Hauptfunktion
  \item \textit{Vorbedingung:} Use Case Check Input wurde erfolgreich durchgef\"uhrt.
  \item \textit{Nachbedingung:} Die Berechnung wird durchgef\"uhrt.
  \item \textit{Nachbedingung im Fehlerfall:} Die Applikation zeigt eine Fehlermeldung an und markiert die Stelle, an der sich der Fehler befindet.
  \item \textit{Hauptakteure:} System
  \item \textit{Nebenakteure:}
  \item \textit{Ausl\"oser:} Der Nutzer hat auf 'run' gedr\"uckt und UC Check Input war erfolgreich.
  \item \textit{Standardablauf:}
    \begin{enumerate}
    \item Das System pr\"uft, ob weniger als 2 Variablen vorhanden sind.
    \item Das System pr\"uft, ob die Randfunktionen g\"ultig sind.
  \end{enumerate}
  \item \textit{Verzweigungen:}
    \begin{enumerate}[label=(1a\arabic*)]
	\item Das System zeigt eine Fehlermeldung, falls mehr als 2 Variablen vorhanden sind.
	\item Die Berechnung wird abgebrochen.
	\end{enumerate}
	 \begin{enumerate}[label=(2a\arabic*)]
	\item Das System zeigt eine Fehlermeldung, falls die Randfunktion ung\"ultig ist.
	\item Die Berechnung wird abgebrochen.
    \end{enumerate}
  \end{itemize}