\textbf{Numerische Parameter \"Andern}
  \begin{itemize}
  \item \textit{Ziel:} Der Nutzer will einen  numerischen Parameter \"andern.
  \item \textit{Einordnung:} Hauptfunktion
  \item \textit{Vorbedingung:} Die Applikation wurde gestartet und der Tab 'Setting' im MainWindow wird angezeigt.
  \item \textit{Nachbedingung:} Der numerische Parameter wurde ver\"andert.
  \item \textit{Nachbedingung im Fehlerfall:} 
  \item \textit{Hauptakteure:} Nutzer
  \item \textit{Nebenakteure:} System
  \item \textit{Ausl\"oser:} Der Nutzer m\"ochte einen  numerischen Parameter \"andern.
  \item \textit{Standardablauf:}
    \begin{enumerate}
    \item Der Nutzer klickt mit der linken Maustaste auf beliebiges Feld der numerischen Parameter. 
    \item Der Nutzer gibt \"uber die Tastatur den neuen numerischen Wert ein.
  \end{enumerate}
  \end{itemize}