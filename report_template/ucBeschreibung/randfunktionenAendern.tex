\textbf{Randfunktionen \"Andern}
  \begin{itemize}
  \item \textit{Ziel:} Der Nutzer will eine Randfunktion \"andern.
  \item \textit{Einordnung:} Hauptfunktion
  \item \textit{Vorbedingung:} Die Applikation wurde gestartet und der Tab 'Setting' im MainWindow wird angezeigt.
  \item \textit{Nachbedingung:} Die Randfunktion wurde ver\"andert.
  \item \textit{Nachbedingung im Fehlerfall:}  
  \item \textit{Hauptakteure:} Nutzer
  \item \textit{Nebenakteure:} System
  \item \textit{Ausl\"oser:} Der Nutzer m\"ochte eine Randfunktion \"andern.
  \item \textit{Standardablauf:}
    \begin{enumerate}
    \item Der Nutzer klickt mit der linken Maustaste auf beliebiges Feld der Randfunktionen an.
    \item Der Nutzer gibt \"uber die Tastatur die neue Randfunktion ein. 
  \end{enumerate}
  \end{itemize}