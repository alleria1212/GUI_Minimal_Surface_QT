\chapter{Vorwort}
\label{ch:1}

\section{Aufgabenstellung und Struktur des Dokuments}
\label{sec:1.1}

Sehr geehrte Damen und Herren, \\

\noindent aufgrund der Notwendigkeit zur Reduzierung von Materialverbrauch, aber auch aus dem Wunsch heraus, \"asthetisch ansprechende Produkte zu produzieren, ben\"otigt unser Unternehmen eine Simulationssoftware zur Berechnung von Minimalfl\"achen. \\ 

\noindent Ausschlaggebend f\"ur eine gute Integration der Software in unseren Arbeitsprozess ist, dass mithilfe einer graphischen Benutzerschnittstelle Randbedingungen und numerische Parameter eingestellt, sowie erstellte Konfigurationen als Datei gespeichert und auch wieder eingelesen werden k\"onnen. Au\ss erdem soll eine Visualisierung der resultierenden Minimalfl\"ache m\"oglich sein. \\

\noindent Wir freuen uns auf eine Zusammenarbeit.\\
Mit freundlichen Gr\"u\ss en \\

\noindent (Jens Deussen und Uwe Naumann)

\section{Projektmanagement}
\label{sec:1.2}

Tom Witter:
\begin{itemize}
\item Implementation Diskretisierung
\item Implementation User Interface
\item Benutzeranforderungen
\end{itemize}
Stefan Jeske:
\begin{itemize}
\item Implementation Controller
\item Implementation User Interface
\item Kompilierung auf dem Cluster
\end{itemize}
Daniel Partida:
\begin{itemize}
\item Implementation Fehlermeldungen im MainWindow
\item Aktivit\"aetsdisgramme
\end{itemize}
Chun-Kan Chow:
\begin{itemize}
\item Implementation der Funktionalit\"aten (Slots, Signale)
\item Implemenation Controller/MainWindow (Laden, Speichern)
\item Benutzerdokumentation
\end{itemize}
Alle:
\begin{itemize}
\item UC
\end{itemize}

\section{Lob und Kritik}
\label{sec:1.3}

Danke Naumann f\"ur die Organisation des Projektes und die gewonnene Praxiserfahrung. ;-)

